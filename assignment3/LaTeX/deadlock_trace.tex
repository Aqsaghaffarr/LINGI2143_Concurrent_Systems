\section{Deadlock trace}
\label{sec:Deadlock trace}

\lstset{
    frame=single,
    breaklines=true
}

Here is the deadlock trace :

\begin{lstlisting}
FALSE
  state 0: {clerks free}*2 {coroners free} {inspectors free} {officers free}*2 {vehicle free}
  -{crime call}->
  state 1: {clerks free} {coroners free} crime {inspectors free} {officers free}*2 {vehicle free}
  -{crime call}->
  state 2: {coroners free} crime*2 {inspectors free} {officers free}*2 {vehicle free}
  -{form patrol}->
  state 3: {coroners free} crime*2 {inspectors free} {patrols free}
  -{patrol found2}->
  state 4: {clerks free} {coroners free} crime {inspectors free} intervention2
  -{crime call}->
  state 5: {coroners free} crime*2 {inspectors free} intervention2
  -investigation->
  state 6: L.dead {coroners free} crime*2 {officers free} report2 {vehicle free}
  -L.deadlock->
  state 7: L.dead {coroners free} crime*2 {officers free} report2 {vehicle free}
\end{lstlisting}

This deadlock trace can be explained as follow : \newline

There is 2 \textit{crime calls} that leads the two clerks to be busy waiting for a patrol. Then, a \textit{patrol} (2 officers and a vehicle) is created ; a clerk found it and send it on the \textit{intervention2} (one clerck is thus free). A \textit{crime call} happens, the only clerck left is now busy. The \textit{coroner}, the \textit{inspector} and the \textit{patrol} goes on investigation. When it is finished, they need a clerck for the report, but the two existing are already busy with a \textit{crime call}, waiting for a patrol (the only one existing is busy, waiting for a clerck to be available). \newline
